\documentclass{beamer}

\usepackage{comment}
\usepackage{color}
\usepackage{listings}
\usepackage{verbatim}
\usepackage{multicol}
\usepackage{booktabs}
\definecolor{green}{RGB}{0,128,0}

\def\EQ#1\EN{\begin{equation*}#1\end{equation*}}
\def\BA#1\EA{\begin{align*}#1\end{align*}}
\def\BS#1\ES{\begin{split*}#1\end{split*}}
\newcommand{\bc}{\begin{center}}
\newcommand{\ec}{\end{center}}
\newcommand{\eq}{\ =\ }
\newcommand{\degc}{$^\circ$C}

\def\p{\partial}
\def\qbs{\boldsymbol{q}}
\def\Dbs{\boldsymbol{D}}
\def\A{\mathcal A}
\def\gC{\mathcal C}
\def\gD{\mathcal D}
\def\gL{\mathcal L}
\def\M{\mathcal M}
\def\P{\mathcal P}
\def\Q{\mathcal Q}
\def\gR{\mathcal R}
\def\gS{\mathcal S}
\def\X{\mathcal X}
\def\bnabla{\boldsymbol{\nabla}}
\def\bnu{\boldsymbol{\nu}}
\renewcommand{\a}{{\alpha}}
%\renewcommand{\a}{{}}
\newcommand{\s}{{\sigma}}
\newcommand{\bq}{\boldsymbol{q}}
\newcommand{\bz}{\boldsymbol{z}}
\def\bPsi{\boldsymbol{\Psi}}

\def\Li{\textit{L}}
\def\Fb{\textbf{f}}
\def\Jb{\textbf{J}}
\def\cb{\textbf{c}}

\def\Dt{\Delta t}
\def\tpdt{{t + \Delta t}}
\def\bpsi{\boldsymbol{\psi}}
\def\dbpsi{\delta \boldsymbol{\psi}}
\def\bc{\textbf{c}}
\def\dbc{\delta \textbf{c}}
\def\arrows{\rightleftharpoons}

\newcommand{\bGamma}{\boldsymbol{\Gamma}}
\newcommand{\bOmega}{\boldsymbol{\Omega}}
%\newcommand{\bPsi}{\boldsymbol{\Psi}}
%\newcommand{\bpsi}{\boldsymbol{\psi}}
\newcommand{\bO}{\boldsymbol{O}}
%\newcommand{\bnu}{\boldsymbol{\nu}}
\newcommand{\bdS}{\boldsymbol{dS}}
\newcommand{\bg}{\boldsymbol{g}}
\newcommand{\bk}{\boldsymbol{k}}
%\newcommand{\bq}{\boldsymbol{q}}
\newcommand{\br}{\boldsymbol{r}}
\newcommand{\bR}{\boldsymbol{R}}
\newcommand{\bS}{\boldsymbol{S}}
\newcommand{\bu}{\boldsymbol{u}}
\newcommand{\bv}{\boldsymbol{v}}
%\newcommand{\bz}{\boldsymbol{z}}
\newcommand{\pressure}{P}

\def\water{H$_2$O}
\def\calcium{Ca$^{2+}$}
\def\copper{Cu$^{2+}$}
\def\magnesium{Mg$^{2+}$}
\def\sodium{Na$^+$}
\def\potassium{K$^+$}
\def\uranium{UO$_2^{2+}$}
\def\hion{H$^+$}
\def\hydroxide{0H$^-$}
\def\bicarbonate{HCO$_3^-$}
\def\carbonate{CO$_3^{2-}$}
\def\cotwo{CO$_2$(aq)}
\def\chloride{Cl$^-$}
\def\fluoride{F$^-$}
\def\phosphoricacid{HPO$_4^{2-}$}
\def\nitrate{NO$_3^-$}
\def\sulfate{SO$_4^{2-}$}
\def\souotwooh{$>$SOUO$_2$OH}
\def\sohuotwocothree{$>$SOHUO$_2$CO$_3$}
\def\soh{$>$SOH}

\newcommand\gehcomment[1]{{{\color{orange} #1}}}
\newcommand\add[1]{{{\color{blue} #1}}}
\newcommand\remove[1]{\sout{{\color{red} #1}}}
\newcommand\codecomment[1]{{{\color{green} #1}}}
\newcommand\redcomment[1]{{{\color{red} #1}}}
\newcommand\bluecomment[1]{{{\color{blue} #1}}}
\newcommand\greencomment[1]{{{\color{green} #1}}}
\newcommand\magentacomment[1]{{{\color{magenta} #1}}}

\begin{comment}
\tiny
\scriptsize
\footnotesize
\small
\normalsize
\large
\Large
\LARGE
\huge
\Huge
\end{comment}

\begin{document}
\title{2D Copper Leaching Problem}
\author{Peter C. Lichtner}
\date{\today}

%\frame{\titlepage}

%-----------------------------------------------------------------------------
\section{\bf Description of 2D Copper Leaching Problem}

\subsection{\bf 2D Copper Leaching Model}

\frame{\frametitle{\bf Description of 2D Copper Leaching Problem}
The ``2D Copper Leaching Problem'' simulates groundwater flow and solute transport for a quarter symmetry element of a 5-spot domain:
\begin{itemize}
  \item Problem domain: $16 \times 16 \times 128$ m$^3$ ($x \times y \times z$)
  \item Grid resolution $0.5 \times 0.5 \times 128$ m$^3$ ($32 \times 32 \times 1$ cells)
  \item Maximum time step size: 0.01 y
  \item Total simulation time: 2 years
  \item Fully saturated, isothermal
  \item 12 primary species
  \item Transport DOFs = 12,288
  
  \begin{tabular}{ll}
  Scaling: & MacBook Pro\\
  \midrule
  1p&7.76 min\\
  2p&4.52 min\\
  4p&3.85 min 
  \end{tabular}
\end{itemize}

}

%-----------------------------------------------------------------------------
\frame{
\frametitle{\bf 2D 5-Spot Velocity Field}
\begin{center}
\includegraphics[width=0.75\linewidth]{./32x32x1.jpg}
\end{center}
}

%-----------------------------------------------------------------------------
\subsection{Flow Governing Equations}

\frame{\frametitle{\bf Flow Equations}

{\bf Continuity Equation:}
\EQ
\frac{\p}{\p t} \big(\varphi \rho\big) + \bnabla\cdot\rho\bq \eq Q
\EN


{\bf Darcy's Law:}
\EQ
\bq \eq -\frac{k}{\mu} \bnabla\big(\pressure-\rho g z\big)
\EN

\normalsize

\begin{columns}[c]
\column{0.5\linewidth}
\begin{itemize}
\item $\varphi \eq $ porosity
%\item $s \eq $ saturation
\item $\rho \eq $ water density
\item $\bk \eq $ intrinsic permeability tensor
%\item $k_r \eq $ relative permeability
\item $\mu \eq $ viscosity
\end{itemize}
\column{0.5\linewidth}
\begin{itemize}
\item $\pressure \eq $ water pressure
\item $g \eq $ gravity
\item $z \eq $ distance in direction of gravity
\item $Q \eq $ source/sink
\end{itemize}
\end{columns}

}

%-----------------------------------------------------------------------------
\frame{\frametitle{\bf Flow Boundary Conditions}

%\begin{itemize}
%\item Dirichlet (e.g. specified pressure)
%\item Neumann (e.g. specified flux)
%\end{itemize}

\begin{itemize}
\item No flow at all boundaries (default)
\item Source/Sink
%\item Hydrostatic
%\item Seepage Face
\end{itemize}

}

%-----------------------------------------------------------------------------
\subsection{Transport Governing Equations}

\frame{\frametitle{\bf Reactive Transport Equations}

\Large

\EQ\label{trans}
\frac{\p}{\p t} \left(\varphi \Psi_j\right) + \bnabla\cdot\bOmega_j \eq Q_j - \sum_m \nu_{jm} I_m
\EN

\EQ\label{flux}
\bOmega_j \eq \big(\bq - \varphi \Dbs\bnabla\big) \Psi_j
\EN

%\bigskip
%\normalsize
\footnotesize
\begin{align*}
\varphi &\eq \text{porosity}\\
\Psi_j &\eq \text{solute concentration for aqueous primary species } j\\
Q_j &\eq \text{source/sink term for aqueous primary species } j\\
\bOmega_j &\eq \text{solute flux for $j$th primary species}; \quad
\bq \eq \text{Darcy velocity} \\
\Dbs &\eq \text{hydrodynamic dispersion} \eq\gD^\ast + \alpha_L|\bnu|\\
\gD^\ast &\eq \text{species {\color{red} independent} coefficient of diffusion}\\
\alpha_L &\eq \text{longitudinal dispersity}\\
\bnu &\eq \text{pore water velocity} \eq \bq/\varphi\\
I_m &\eq \text{mineral reaction rate}\\
\nu_{jm} &\eq \text{stoichiometric coefficient}
\end{align*}

}

%-----------------------------------------------------------------------------
\frame{\frametitle{\bf Mineral Kinetic Rate Law}

\begin{itemize}
\item Transition State Theory (TST) Rate Law:
\EQ
\sum_j\nu_{jm}\A_j \arrows \M_m
\EN
\EQ
I_m \eq -a_m \Big(\sum_l k_{ml} \P_{ml}\Big) \left(1-K_m Q_m\right) \zeta_m
\EN
\EQ
Q_m \eq \prod_j\big(\gamma_j m_j\big)^{\nu_{jm}}
\EN
\EQ\label{prefactor}
\P_{ml} \eq \prod_i\dfrac{\big(\gamma_i m_i\big)^{\a_{il}^m}}{1+K_{ml}\big(\gamma_i m_i\big)^{\beta_{il}^m} }
\EN
\EQ
\zeta_m \eq
\left\{
\begin{array}{ll}
1, \ \varphi_m>0 \ \text{or} \ K_m Q_m > 1\\
0, \ \text{otherwise}
\end{array}
\right.
\EN
\end{itemize}

}

%-----------------------------------------------------------------------------
\frame{\frametitle{\bf Transport Boundary Conditions}

\begin{itemize}
%\item Dirichlet (e.g. specified concentration)
%\item Neumann (e.g. specified mass flux)
%\item Zero Gradient (e.g. outflow boundary)
\item Closed system (no flow)
\end{itemize}


}

%-----------------------------------------------------------------------------
\frame{\frametitle{\bf Transport - Averaging Schemes}
\Large
\EQ\label{flux}
\bOmega_j \eq \big(\bq - \varphi\Dbs\bnabla\big) \Psi_j
\EN

\bigskip
%\normalsize
\begin{itemize}
%\item $s$: arithmetic
\item $\bq \Psi_j$: upwinding
\item $\varphi D$: harmonic
%\item $D$: harmonic
\end{itemize}
}

%-----------------------------------------------------------------------------
\section{Description of Input Deck}

\subsection{MODE}
\begin{frame}[fragile,containsverbatim]\frametitle{\bf MODE}

\begin{semiverbatim}

MODE RICHARDS

CHEMISTRY
\end{semiverbatim}

\begin{itemize}
  \item Single-phase fully saturated flow
  \item Flow coupled to multicomponent reactive transport
\end{itemize}

\end{frame}

%-----------------------------------------------------------------------------
\begin{frame}[fragile,containsverbatim]\frametitle{\bf GRID}

\begin{itemize}
  \item Problem domain: $16 \times 16 \times 128$ m ($x \times y \times z$)
  \item Grid resolution $0.5 \times 0.5 \times 128$ m$^3$
\end{itemize}

\begin{semiverbatim}
GRID
  TYPE structured
  NXYZ 32 32 1
  BOUNDS
    0.d0 0.d0 0.d0
    16.d0 16.d0 128.d0
  /
END
\end{semiverbatim}

\end{frame}

%-----------------------------------------------------------------------------
\subsection{REGION}

\begin{frame}[fragile,containsverbatim,allowframebreaks]\frametitle{REGION}

\begin{itemize}
  \item Delineate regions in the 3D domain for:
  \begin{itemize}
    \item entire domain
    \item west boundary face
    \item east boundary face
    \item top boundary face
    \item well screens
  \end{itemize}
\end{itemize}

\begin{semiverbatim}
REGION all
  COORDINATES
    0.d0 0.d0 0.d0
    16.d0 16.d0 128.d0
  /
END

\newpage
REGION layer2        \bluecomment{! layer of domain}
  COORDINATES
    0.d0 0.d0 30.d0
    5000.d0 2500.d0 50.d0
  /
END

REGION Top           \bluecomment{! top surface}
  COORDINATES
    0.d0 0.d0 100.d0
    5000.d0 2500.d0 100.d0
  /
  FACE TOP
END

\newpage
REGION Extraction_well
  COORDINATES             \bluecomment{! vertical line}
    3750.d0 1250.d0 20.d0
    3750.d0 1250.d0 55.d0
  /
END

REGION Obs_pt_center      \bluecomment{! point}
  \magentacomment{COORDINATE} 2500.d0 1250.d0 60.d0
END

\end{semiverbatim}

\end{frame}


%-----------------------------------------------------------------------------
\subsection{MATERIAL\_PROPERTY}

\begin{frame}[fragile,containsverbatim]\frametitle{\bf MATERIAL\_PROPERTY}

\begin{itemize}
  \item Isotropic permeability
  \item Fully saturated
\end{itemize}

\begin{semiverbatim}
MATERIAL_PROPERTY oxide-ore
  ID 1
  POROSITY 0.05d0
  TORTUOSITY 1.d0
  PERMEABILITY
    PERM_ISO 1.5d-13 \bluecomment{! isotropic permeability tensor}
  /
  SATURATION_FUNCTION default
END
\end{semiverbatim}

\end{frame}

%-----------------------------------------------------------------------------
\subsection{SATURATION\_FUNCTION}

\begin{frame}[fragile]\frametitle{\bf SATURATION\_FUNCTION}

%\begin{itemize}
%\item Add van Genuchten parameters
%\item Assume Mualem permeability (default)
%\end{itemize}

\begin{semiverbatim}
SATURATION_FUNCTION default
END
\end{semiverbatim}

\end{frame}

%-----------------------------------------------------------------------------
\subsection{\bf FLUID\_PROPERTY}

\begin{frame}[fragile,containsverbatim]\frametitle{\bf FLUID\_PROPERTY}

\begin{itemize}
  \item Assign molecular diffusion coefficient of $10^{-9}$ m$^2$/s to all aqueous species
\end{itemize}

\begin{semiverbatim}

FLUID_PROPERTY
  DIFFUSION_COEFFICIENT 1.d-9   \bluecomment{! [m^2/s]}
END
\end{semiverbatim}

\end{frame}

%-----------------------------------------------------------------------------
\subsection{CHEMISTRY}

\begin{frame}[fragile,allowframebreaks]\frametitle{\bf CHEMISTRY}

\begin{semiverbatim}

CHEMISTRY
  PRIMARY_SPECIES
  /
  SECONDARY_SPECIES
  /
  GAS_SPECIES
  /
  MINERALS
  /
  MINERAL_KINETICS
  /
END
\end{semiverbatim}

\newpage
~

\begin{itemize}

\item 12 primary species including redox reaction for Cu$^+$--Cu$^{2+}$
\item Species properties (log$K_i$, $\nu_{ji}$, $a_i^\circ$, $z_i$, $W_i$, $\overline V_m$, $\cdots$) read from thermodynamic database.
\item Species names must match database entry
\end{itemize}
\begin{columns}[c]
    \column{0.5\linewidth}
\begin{semiverbatim}
  PRIMARY_SPECIES
    Na+
    K+
    Ca++
    H+
    Cu++
    Al+++
    Fe++
\end{semiverbatim}
        \column{0.5\linewidth}
\begin{semiverbatim}
    SiO2(aq)
    HCO3-
    SO4--
    Cl-
    O2(aq)
  /
\end{semiverbatim}
  \end{columns}

\newpage
~
\begin{itemize}
\item User responsible for selecting aqueous secondary species
\item Secondary species assumed to obey local equilibrium conditions
and do not add to number of degrees of freedom
\end{itemize}
\vspace{-5mm}
  \begin{columns}[c]
    \column{0.333\linewidth}
\begin{semiverbatim}
  SECONDARY_SPECIES
    OH-
    CO3--
    CO2(aq)
    CaOH+
    CaHCO3+
    CaCO3(aq)
    CaSO4(aq)
    Cu+
    CuOH+
    CuO2--
    CuCl+
    CuCl2(aq)
\end{semiverbatim}
        \column{0.333\linewidth}
\begin{semiverbatim}
    CuCl2-
    CuCl3--
    CuCl4--
    CuSO4(aq)
    HSO4-
    AlOH++
    Al(OH)2+
    Al(OH)3(aq)
    Al(OH)4-
    Al(SO4)2-
    AlSO4+
\end{semiverbatim}
        \column{0.333\linewidth}
\begin{semiverbatim}
    Fe+++
    Fe(OH)2(aq)
    Fe(OH)2+
    Fe(OH)3(aq)
    Fe(OH)3-
    Fe(OH)4-
    FeSO4(aq)
    FeSO4+
    Fe(SO4)2-
  /
\end{semiverbatim}
  \end{columns}
  
  \newpage
  ~
  \begin{itemize}
  \item Gas species O$_{\rm 2(aq)}$ needed for redox reactions
  \end{itemize}
\begin{semiverbatim}

  GAS_SPECIES
    CO2(g)
    O2(g)
  /
\end{semiverbatim}
  
  \newpage
  ~
  \begin{itemize}
  \item Mineral list includes both active primary and secondary minerals and passive mineral speciation constraints and saturation indices
  \end{itemize}
  \vspace{-3mm}
  \begin{columns}[c]
    \column{0.5\linewidth}
\begin{semiverbatim}
  MINERALS
    Albite
    Alunite
    Anorthite
    Antlerite
    Aragonite
    Brochantite
    Calcite
    Chalcanthite
    Chalcedony
    Chrysocolla2
    Cuprite
    
\end{semiverbatim}
    \column{0.5\linewidth}
\begin{semiverbatim}
    Gibbsite
    Goethite
    Gypsum
    Jarosite
    Jurbanite
    K-Feldspar
    Kaolinite
    Malachite
    Muscovite
    SiO2(am)
    Tenorite
    Quartz
  /
\end{semiverbatim}
  \end{columns}

  \newpage
  \begin{itemize}
  \item Specify rate constants, prefactors, surface area powers
  \end{itemize}
\begin{semiverbatim}
  MINERAL_KINETICS
    Alunite
      RATE_CONSTANT 1.d-11
    /
    Chrysocolla2
      SURFACE_AREA_VOL_FRAC_POWER 0.666666667d0
      PREFACTOR
        RATE_CONSTANT 1.d-10
        PREFACTOR_SPECIES H+
          ALPHA 0.39
        /
      /
    /
    
    
    Goethite
      SURFACE_AREA_VOL_FRAC_POWER 0.666666667d0
      RATE_CONSTANT 1.d-11
    /
    Gypsum
      RATE_CONSTANT 1.d-10
    /
    Jarosite
      RATE_CONSTANT 1.d-11
    /
    Jurbanite
      RATE_CONSTANT 1.d-11
    /
    Kaolinite
      SURFACE_AREA_VOL_FRAC_POWER 0.666666667d0
      RATE_CONSTANT 1.d-13
    /
    Muscovite
      SURFACE_AREA_VOL_FRAC_POWER 0.666666667d0
      RATE_CONSTANT 1.d-13
    /
    SiO2(am)
      RATE_CONSTANT 1.d-11
    /
    Quartz
      SURFACE_AREA_VOL_FRAC_POWER 0.666666667d0
      RATE_CONSTANT 1.d-14
    /
  /
\end{semiverbatim}

\end{frame}

%-----------------------------------------------------------------------------
\subsection{FLOW\_CONDITION}

\begin{frame}[fragile,allowframebreaks]\frametitle{FLOW\_CONDITION}

\begin{semiverbatim}

FLOW_CONDITION initial
  TYPE
    PRESSURE hydrostatic  \bluecomment{! hydrostatic condition}
  /
  DATUM 0.d0 0.d0 90.d0   \bluecomment{! point in space}
  GRADIENT
    PRESSURE -0.002 0. 0. \bluecomment{! gradient for pressure \redcomment{[m/m]}}
  /                       \bluecomment{!   unless \redcomment{dz} specified \redcomment{[Pa/m]}}
  PRESSURE 101325.d0      \bluecomment{! pressure at datum}
END

\newpage

FLOW_CONDITION river
  TYPE
    PRESSURE seepage     \bluecomment{! seepage face condition}
  /
  INTERPOLATION linear   \bluecomment{! dataset time interpolation}
  CYCLIC                 \bluecomment{! cycle dataset}
  DATUM FILE river_stage.txt  \bluecomment{! read transient datum}
  PRESSURE 101325.d0          \bluecomment{!    dataset from file}
END

\newpage
FLOW_CONDITION injection
  TYPE                           \bluecomment{! volumetric flow}
    RATE scaled_volumetric_rate  \bluecomment{!   rate scaled by}
  /                              \bluecomment{!   f(permeability)}
  RATE 1.d5 m^3/day  \bluecomment{! flow rate 10,000 [m^3/day]}
END    

FLOW_CONDITION extraction
  TYPE
    RATE scaled_volumetric_rate
  /
  RATE -1.d5 m^3/day  \bluecomment{! flow rate -10,000 [m^3/day]}
END                   \bluecomment{! negative rate = extraction}

\end{semiverbatim}
\end{frame}

%-----------------------------------------------------------------------------
\subsection{TRANSPORT\_CONDITION / CONSTRAINT}

\begin{frame}[fragile,allowframebreaks]\frametitle{\bf TRANSPORT\_CONDITION / CONSTRAINT}

\begin{itemize}
  \item Set up two transport constraints for primary species and minerals corresponding to initial conditions and injected fluid composition
   \begin{itemize}
     \item Primary species concentrations
     \item Reactive minerals---both primary and secondary
   \end{itemize}
\end{itemize}

\newpage

\begin{semiverbatim}

CONSTRAINT initial
  CONCENTRATIONS
    Na+        5.0d-3   T
    K+         2.5d-5   M Muscovite
    Ca++       6.5d-4   M Calcite
    H+         8.d0    pH
    Cu++       6.4d-9   M Chrysocolla2
    Al+++      2.8d-17  M Kaolinite
    Fe++       1.2d-23  M Goethite
    SiO2(aq)   1.8d-4   M Chalcedony
    HCO3-      -3.d0    G CO2(g)
    SO4--      5.0d-4   T
    Cl-        3.7d-3   Z
    O2(aq)     -0.699d0 G O2(g)
  /
\end{semiverbatim}

  \newpage
\begin{semiverbatim}

  MINERALS !volume fraction specific surface area
    Alunite       0.d0    1.d0
    Chrysocolla2  5.0d-3  1.d0
    Goethite      2.5d-2  1.d0
    Gypsum        0.d0    1.d0
    Jarosite      0.d0    1.d0
    Jurbanite     0.d0    1.d0
    Kaolinite     5.0d-2  1.d0
    Muscovite     5.0d-2  1.d0
    SiO2(am)      0.d0    1.d0
    Quartz        8.2d-1  1.d0
  /
END
\end{semiverbatim}

\newpage
\begin{semiverbatim}
CONSTRAINT inlet
  CONCENTRATIONS
    Na+        5.0d-3   T
    K+         1.3d-4   M Jarosite
    Ca++       1.1d-2   M Gypsum
    H+         1.d0    pH
    Cu++       1.0d-8   T 
    Al+++      2.5d-2   T 
    Fe++       3.4d-9   M Goethite
    SiO2(aq)   1.9d-3   M SiO2(am)
    HCO3-      -2.d0    G CO2(g)
    SO4--      6.1d-2   Z
    Cl-        5.0d-3   T
    O2(aq)     -0.699d0 G O2(g)
  /
END
\end{semiverbatim}

\end{frame}

%-----------------------------------------------------------------------------
\subsection{STRATA}

\begin{frame}[fragile]\frametitle{\bf STRATA}

\begin{itemize}
\item Couple material types with regions
\end{itemize}

\begin{semiverbatim}
STRATA
  REGION layer1
  MATERIAL soil1
END

STRATA
  REGION layer2
  MATERIAL soil2
END

\ldots

STRATA
  REGION layer4
  MATERIAL soil4
END
\end{semiverbatim}

\end{frame}


%-----------------------------------------------------------------------------
\subsection{OBSERVATION}

\begin{frame}[fragile]\frametitle{\bf OBSERVATION}

\begin{itemize}
\item Couple observation points with regions in model
\end{itemize}

\begin{semiverbatim}
OBSERVATION  \bluecomment{! observation point assigned region name}
  REGION Obs_pt_center  \bluecomment{! region name}
  AT_CELL_CENTER        \bluecomment{! do not interpolate}
  VELOCITY              \bluecomment{! include velocity}
END

OBSERVATION
  REGION Obs_pt_west
  AT_CELL_CENTER
  VELOCITY
END

\dots
\end{semiverbatim}

\end{frame}

%-----------------------------------------------------------------------------
\subsection{NEWTON\_SOLVER}

\begin{frame}[fragile]\frametitle{\bf NEWTON\_SOLVER}

\begin{itemize}
  \item Set converged if maximum pressure change within all cells is less than 1 Pa.
\end{itemize}

\begin{semiverbatim}

NEWTON_SOLVER FLOW
ITOL_UPDATE 1.d0   \bluecomment{! infinity norm of update vector}
END
\end{semiverbatim}

\end{frame}

%-----------------------------------------------------------------------------
\subsection{INITIAL\_CONDITION}

\begin{frame}[fragile]\frametitle{\bf INITIAL\_CONDITION}

\begin{itemize}
\item Couple the greencomment{initial} flow and transport conditions with region \greencomment{all} for the initial condition
\end{itemize}

\begin{semiverbatim}

INITIAL_CONDITION
  FLOW_CONDITION initial
  TRANSPORT_CONDITION initial
  REGION all
END

\end{semiverbatim}

\end{frame}

%-----------------------------------------------------------------------------
\subsection{BOUNDARY\_CONDITION}

\begin{frame}[fragile,allowframebreaks]\frametitle{BOUNDARY\_CONDITION}

\small
\begin{semiverbatim}
BOUNDARY_CONDITION west
  FLOW_CONDITION initial
  TRANSPORT_CONDITION west
  REGION west
END

BOUNDARY_CONDITION east
  FLOW_CONDITION river
  TRANSPORT_CONDITION initial
  REGION east
END

BOUNDARY_CONDITION top
  FLOW_CONDITION recharge
  TRANSPORT_CONDITION initial
  REGION top
END
\end{semiverbatim}

\end{frame}

%-----------------------------------------------------------------------------
\subsection{SOURCE\_SINK (CONDITION)}

\begin{frame}[fragile,allowframebreaks]\frametitle{\bf SOURCE\_SINK}

\begin{semiverbatim}
SOURCE_SINK injection_well   \bluecomment{! source/sink name (optional)}
  FLOW_CONDITION injection       \bluecomment{! flow condition name}
  TRANSPORT_CONDITION injection  \bluecomment{! tran. condition name}
  REGION injection_well        \bluecomment{! location of source/sink}
END

SOURCE_SINK extraction_well
  FLOW_CONDITION extraction
  TRANSPORT_CONDITION initial
  REGION extraction_well
END
\end{semiverbatim}

\end{frame}

%-----------------------------------------------------------------------------
\subsection{TIME}

\begin{frame}[fragile]\frametitle{\bf TIME}

\begin{itemize}
\item Set final simulation time to 10 years
\item Set initial time step size to 1.e-2 days
\item Set maximum time step size to 0.1 years
\end{itemize}


\begin{semiverbatim}

TIME
  FINAL_TIME 10.d0 y
  INITIAL_TIMESTEP_SIZE 1.d-2 d
  MAXIMUM_TIMESTEP_SIZE 0.1 y    \bluecomment{! ensures CFL ~<= 1.}
END
\end{semiverbatim}

\end{frame}

%-----------------------------------------------------------------------------
\subsection{OUTPUT}

\begin{frame}[fragile]\frametitle{\bf OUTPUT}

\begin{itemize}
\item Print entire solution every year in Tecplot block datapacking and PFLOTRAN HDF5 format compatible with Visit
\item Print entire solution at 1.25, 1.5 and 1.75 years to see fluctuation in river stage
\item Print solution at observation points every time step
\item Include velocities when entire solution is printed
\end{itemize}

\begin{semiverbatim}

OUTPUT
  TIMES y 1.25 1.5 1.75  \bluecomment{! specific times}
  PERIODIC TIME 1. y     \bluecomment{! solution every year}
  PERIODIC_OBSERVATION TIMESTEP 1  \bluecomment{! observ. every step}
  FORMAT TECPLOT BLOCK   \bluecomment{! Tecplot BLOCK format}
  FORMAT HDF5            \bluecomment{! PFLOTRAN HDF5 Visit format}
  PRINT_COLUMN_IDS       \bluecomment{! Adds column ids to obs. header}
  VELOCITIES             \bluecomment{! include velocities}
END
\end{semiverbatim}

\end{frame}

%-----------------------------------------------------------------------------
\section{CHECKPOINT/RESTART}

\begin{frame}[fragile]\frametitle{\bf CHECKPOINT/RESTART}

\begin{itemize}
\item Checkpointing: saving the ``state'' of the simulation at a point in time, from which the entire solution can be reconstructed.
  \begin{itemize}
    \item Fault tolerance
    \item Restarting from a saved initial condition
    \item Checkpoint files named ``pflotran.chk\#'' where \# is the time step number (e.g. pflotran.chk100)
    \item Restart file named ``restart.chk''
  \end{itemize}
\item Restart: resetting a simulation to a ``state''
\end{itemize}


\begin{semiverbatim}

CHECKPOINT 100          \bluecomment{! write a checkpoint file every}
                        \bluecomment{!   100 time steps}
RESTART pflotran.chk100 \bluecomment{! restart simulation using file}
                        \bluecomment{!   \redcomment{restart.chk}}
RESTART restart.chk \redcomment{0.d0}  \bluecomment{! reset time to 0.}
\end{semiverbatim}

\end{frame}

\end{document} 
