\documentclass{beamer}

\usepackage{comment}
\usepackage{color}
\usepackage{listings}
\usepackage{verbatim}
\usepackage{multicol}
\usepackage{booktabs}
\definecolor{green}{RGB}{0,128,0}

\def\EQ#1\EN{\begin{equation*}#1\end{equation*}}
\def\BA#1\EA{\begin{align*}#1\end{align*}}
\def\BS#1\ES{\begin{split*}#1\end{split*}}
\newcommand{\bc}{\begin{center}}
\newcommand{\ec}{\end{center}}
\newcommand{\eq}{\ =\ }
\newcommand{\degc}{$^\circ$C}

\def\p{\partial}
\def\qbs{\boldsymbol{q}}
\def\Dbs{\boldsymbol{D}}
\def\A{\mathcal A}
\def\gC{\mathcal C}
\def\gD{\mathcal D}
\def\gL{\mathcal L}
\def\M{\mathcal M}
\def\P{\mathcal P}
\def\Q{\mathcal Q}
\def\gR{\mathcal R}
\def\gS{\mathcal S}
\def\X{\mathcal X}
\def\bnabla{\boldsymbol{\nabla}}
\def\bnu{\boldsymbol{\nu}}
\renewcommand{\a}{{\alpha}}
%\renewcommand{\a}{{}}
\newcommand{\s}{{\sigma}}
\newcommand{\bq}{\boldsymbol{q}}
\newcommand{\bz}{\boldsymbol{z}}
\def\bPsi{\boldsymbol{\Psi}}

\def\Li{\textit{L}}
\def\Fb{\textbf{f}}
\def\Jb{\textbf{J}}
\def\cb{\textbf{c}}

\def\Dt{\Delta t}
\def\tpdt{{t + \Delta t}}
\def\bpsi{\boldsymbol{\psi}}
\def\dbpsi{\delta \boldsymbol{\psi}}
\def\bc{\textbf{c}}
\def\bx{\textbf{x}}
\def\dbc{\delta \textbf{c}}
\def\dbx{\delta \textbf{x}}
\def\arrows{\rightleftharpoons}

\newcommand{\bGamma}{\boldsymbol{\Gamma}}
\newcommand{\bOmega}{\boldsymbol{\Omega}}
%\newcommand{\bPsi}{\boldsymbol{\Psi}}
%\newcommand{\bpsi}{\boldsymbol{\psi}}
\newcommand{\bO}{\boldsymbol{O}}
%\newcommand{\bnu}{\boldsymbol{\nu}}
\newcommand{\bdS}{\boldsymbol{dS}}
\newcommand{\bg}{\boldsymbol{g}}
\newcommand{\bk}{\boldsymbol{k}}
%\newcommand{\bq}{\boldsymbol{q}}
\newcommand{\br}{\boldsymbol{r}}
\newcommand{\bR}{\boldsymbol{R}}
\newcommand{\bS}{\boldsymbol{S}}
\newcommand{\bu}{\boldsymbol{u}}
\newcommand{\bv}{\boldsymbol{v}}
%\newcommand{\bz}{\boldsymbol{z}}
\newcommand{\pressure}{P}

\newcommand\gehcomment[1]{{{\color{orange} #1}}}
\newcommand\add[1]{{{\color{blue} #1}}}
\newcommand\remove[1]{\sout{{\color{red} #1}}}
\newcommand\codecomment[1]{{{\color{green} #1}}}
\newcommand\redcomment[1]{{{\color{red} #1}}}
\newcommand\bluecomment[1]{{{\color{blue} #1}}}
\newcommand\greencomment[1]{{{\color{green} #1}}}
\newcommand\magentacomment[1]{{{\color{magenta} #1}}}

\begin{comment}
\tiny
\scriptsize
\footnotesize
\small
\normalsize
\large
\Large
\LARGE
\huge
\Huge
\end{comment}

\begin{document}
\title{CLM-CN Reaction\ldots in a Nutshell}
\author{Glenn Hammond}
\date{\today}

%\frame{\titlepage}

%-----------------------------------------------------------------------------
\section{Description of CLM-CN Reaction}

\subsection{Batch CLM-CN Reaction Conceptual Model}

\frame{\frametitle{Schematic of CLM-CN Reaction Network}
\includegraphics[width=\linewidth]{./CLM-CN_cycle}
}

%-----------------------------------------------------------------------------
\subsection{Governing Equations}

\frame{\frametitle{Reaction Expression}

\Large

\EQ\label{CN_rxn}
\text{CN}_u \eq \left(1-f\right) \text{CN}_d + f \text{CO}_2 + n \text{N}_\text{mineral}
\EN

\EQ\label{n_calc}
n \eq u - \left(1-f\right) d
\EN

\footnotesize
\BA
\text{CN}_u &\eq \text{upstream carbon pool } [\text{mol C}/m^3] \\
\text{CN}_d &\eq \text{downstream carbon pool } [\text{mol C}/m^3] \\
\text{CO}_2 &\eq \text{nitrogen concentration } [\text{mol CO}_2/m^3] \\
\text{N}_\text{mineral} &\eq \text{nitrogen concentration } [\text{mol N}/m^3] \\
u &\eq \text{\text{C}/\text{N} atomic weight ratio divided by upstream \text{C}/\text{N} ratio} \\
d &\eq \text{\text{C}/\text{N} atomic weight ratio divided by downstream \text{C}/\text{N} ratio} \\
f &\eq \text{respiration fraction} \\
\EA
}

\frame{\frametitle{Kinetic Rate Expression}

\Large

\EQ\label{CN_kinetic_rxn}
rate \eq f_T f_\theta f_\text{pi} k \text{CN}_u
\EN

%\bigskip
%\normalsize
%\footnotesize
\scriptsize
\BA
f_T &\eq \exp\left[308.56 \left(\frac{1}{71.02}-\frac{1}{T - 227.13}\right)\right]\\
f_\theta &\eq \frac{\log\left(\theta_\text{min}/\theta\right)}{\log\left(\theta_\text{min}/\theta_\text{max}\right)}\\
f_\text{pi} &\eq \frac{\text{N}_\text{mineral}}{\text{N}_\text{mineral} + K_{\text{N}_\text{mineral}}} \\
k &\eq \text{kinetic rate constant} [s^{-1}]\\
T &\eq \text{temperature } [K] \\
\theta &\eq \text{moisture content or saturation } [-] \\
\text{CN}_u &\eq \text{upstream carbon pool } [\text{mol C}/m^3] \\
\text{N}_\text{mineral} &\eq \text{nitrogen concentration } [\text{mol N}/m^3] \\
K_\text{N} &\eq \text{nitrogen half saturation constant } [\text{mol N}/m^3]\\
\EA

}

\frame{\frametitle{Kinetic Reaction Equations}


\EQ
\text{CN}_u \eq \left(1-f\right) \text{CN}_d + f \text{CO}_2 + n \text{N}_\text{mineral}
\EN

\EQ
rate \eq f_T f_\theta f_\text{pi} k \text{CN}_u
\EN

\Large
\BA
\frac{\p}{\p t} \left(\text{CN}_u\right) &\eq -rate \\
\frac{\p}{\p t} \left(\text{CN}_d\right) &\eq \left(1-f\right) rate \\
\frac{\p}{\p t} \left(\text{CO}_2\right) &\eq f \cdot rate \\
\frac{\p}{\p t} \left(\text{N}_\text{mineral}\right) &\eq n \cdot rate \\
\EA

}


\frame{\frametitle{Newton-Raphson Method}
\LARGE
\BA
\Fb\left(\bx^{k+1,i}\right) &\eq \frac{\bx^{k+1,i}-\bx^k}{\Dt} - \sum_{irxn} \bnu_{irxn} \cdot rate_{irxn} \\
\\
\Jb \dbx &\eq -\Fb\left(\bx^{k+1,i}\right) \\
\\
\bx^{k+1,i+1} &\eq \bx^{k+1,i} + \dbx
\EA
\small
\BA
k &\eq \text{time step} \\
i &\eq \text{iteration}
\EA
}

%-----------------------------------------------------------------------------
\subsection{PFLOTRAN Input Specification}

\begin{frame}[fragile,allowframebreaks]\frametitle{CHEMISTRY}
\small
\begin{semiverbatim}
CHEMISTRY
  ...
  IMMOBILE_SPECIES
    N
    C
    SOM1
    SOM2
    SOM3
    SOM4
    LabileC
    CelluloseC
    LigninC
    LabileN
    CelluloseN
    LigninN
  /
  ...
\end{semiverbatim}
\newpage
\begin{semiverbatim}

  ...
  REACTION_SANDBOX
    CLM-CN
      POOLS   \bluecomment{! CN ratio}
        SOM1   12.d0   \bluecomment{! -> u or d}
        SOM2   12.d0
        SOM3   10.d0
        SOM4   10.d0
        Labile
        Cellulose
        Lignin
      /
      ...
\end{semiverbatim}
\newpage
\begin{semiverbatim}
      ...
      REACTION
        UPSTREAM_POOL Labile         \bluecomment{! CN_u}
        DOWNSTREAM_POOL SOM1         \bluecomment{! CN_d}
        TURNOVER_TIME 20. h          \bluecomment{! -> k}
        RESPIRATION_FRACTION 0.39d0  \bluecomment{! f}
        N_INHIBITION 1.d-10          \bluecomment{! K_N}
      /
      REACTION
        UPSTREAM_POOL SOM1
        DOWNSTREAM_POOL SOM2
        TURNOVER_TIME 14. d
        RESPIRATION_FRACTION 0.28d0
        N_INHIBITION 1.d-10
      /
      ...
    / \bluecomment{! END CLM-CN}
  / \bluecomment{! END REACTION_SANDBOX}
/ \bluecomment{! END CHEMISTRY}
\end{semiverbatim}
\end{frame}

\begin{frame}[fragile]\frametitle{CONSTRAINT}
%\small
\footnotesize
\begin{semiverbatim}
CONSTRAINT initial
  CONCENTRATIONS   \bluecomment{! moles/L}
    A(aq)  1.d-40  T
  /
  IMMOBILE       \bluecomment{! moles/m^3}
    N     1.d-6
    C     1.d-6
    SOM1  1.d-10 \bluecomment{! moles C/m^3}
    SOM2  1.d-10
    SOM3  1.d-10
    SOM4  1.d-10
    LabileC     0.1852d-3
    CelluloseC  0.4578d-3
    LigninC     0.2662d-3
    LabileN     0.00508954d-3
    CelluloseN  0.01258096d-3
    LigninN     0.00731553d-3
  /
END
\end{semiverbatim}
\end{frame}

\end{document}
