%Wiki home Clone wiki  History Documentation / FortranCodingStandard

\section*{PFLOTRAN Fortran Formatting Protocol}

\subsection*{Guidelines}

\begin{itemize}
\item Free format

\item Use \& for continuation

\item No tabs whatsoever, 2-space indentation, i.e.
\begin{Verbatim}
Level 1
  Level 2
    Level 3
      Level 4
        Level 5
        ...
\end{Verbatim}
\item Maximum source width is 80 characters. Use a continuation line beyond

\begin{Verbatim}
123456789+123456789+123456789+123456789+123456789+123456789+123456789+123456789+
                 print \*, 'This sentence is just barely too long for a single &
                          &line.'
                 call subroutine_with_many_arguments(argument1, argument2, &
                                                     argument3, argument4)
\end{Verbatim}
\item Maximum of 31 characters for subroutine/variable/etc. names. Try to be as concise as possible.

\item Capitalization
\begin{itemize}
\item All fortran syntax should be lower case, i.e. subroutine, module, contains, use, etc.
\item All variable names should be lower case.

\item All module names should be lower case with `\_module` appended.

\item Use PETSc-style capitalization in subroutine/function names, i.e.

VecGetArrayF90, VecDestroy
For example, the following changes should take place:

\begin{Verbatim}
Grid_get_t -> GridGetTime
Grid_setvel -> GridSetVel or GridSetVelocity
Grid_update_dt -> GridUpdateDt or GridUpdateTimestep
\end{Verbatim}

\end{itemize}

\item Pin all module, subroutine, function, and contains declarations up against the left side. 
This leaves more room for indentation later on and is not confusing.
\item The default private/public attribute for modules is `private`
\item `implicit none` at top of every file, subroutine, function, interface
\item `PetscReal` instead of `double precision` or `real*8`
\item `PetscInt` instead of `integer`
\item `PetscBool` instead of `logical`
\item All pointers have `\_p` appended (i.e. `array\_p`)
\item NEVER use PETSc's F77 approach to pointers: a PetscInt/PetscReal array sized to 1 combined with a PetscOffset. 
If you are not sure, ask.
\item No goto's (this may not be possible with legacy portions)

\item User appropriate spacing to improve readability:

if(OneNumber>AnotherNumber.and.ALogical==.true.)then

is better viewed as

if (OneNumber > AnotherNumber .and. ALogical == .true.) then

\item Distinguish between natural, local, and local-ghosted coordinate indices: e.g. na, n, ng, respectively. (GEH: This needs go be revised).

\item For field variables that are time-stepped, repeat the first letter of the variable name to 
distinguish the field at time k+1 from the value at time k. E.g., `density` refers to 
density at time k, and `ddensity` refers to density at time k+1.
\end{itemize}

\subsection*{Filename and Module/Class Naming Convention}

\begin{itemize}
\item Modules and Classes are mixed case with underscores between words and `\_module` (or `\_class` for F03 classes) appended.

\begin{Verbatim}
Reaction_Sandbox_module
Reaction_Sandbox_Base_class
\end{Verbatim}

\item The corresponding filename is the module name with (1) `\_module` or `\_class` removed, (2) all lower case, and (3) `.F90` appended.

\begin{Verbatim}
reaction_sandbox.F90
reaction_sandbox_base.F90
\end{Verbatim}
\end{itemize}

\subsection*{Example Fortran Source Code}

An example source would be (!comment denotes all commentary on example)

\begin{Verbatim}
module Example_module

  implicit none

  private  !comment: all variables/subroutines, etc. are private by default

#include ``whatever.h''

  public :: GridCreate, GridGetTime

  PetscReal, save :: file_global_variable

contains
!************************************************************************** !
!
! GridSetup: Initializes the grid.
! author: John Doe
! date: 01/01/07
!
!************************************************************************** !
subroutine GridSetup(integer_in, real_in)

  use whatever_module

  implicit none

#include "whatever.h"

  PetscInt :: integer_in  !comment: note that the subroutine arguments are
  PetscReal :: real_in      !comment: declared first

  PetscBool :: whatever    !comment: note that declarations are group by type
  PetscInt :: i
  PetscInt :: integer1, integer2
  PetscReal  :: real1, real2
  PetscReal  :: real3, real4
  character(len=MAXWORDLENGTH) :: word
  PetscReal, pointer :: real_p(:)

  ...
  ! use the newer relational operators in logical expressions
  if (grid%ndof >= 2 .and. (.not.logical_whatever .or. &
      integer1 /= integer2)) then
    do i=1,2
      call Whatever
    enddo
  endif

  ! fortran switch
  select case (word)
    case ('flow')
      call Whatever
    case ('transport')
      call Whatever2(argument1, argument2, argument3, argument4, &
                     argument5)
  end select
  ...
  nullify(real_p)

end subroutine GridSetup

!************************************************************************** !
!
! GridGetTime: Returns the current time in the simulation.
! author: John Doe
! date: 01/01/07
!
!************************************************************************** !
PetscReal function GridGetTime(...)

  use another_module

  implicit none

#include "whatever.h"

  PetscInt :: integer1
  PetscReal :: real1
  character(len=MAXWORDLENGTH) :: word

  ...
  ...
  GridGetTime = x

end function GridGetTime

end module Example_module
\end{Verbatim}