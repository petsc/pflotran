
\section{Introduction}

PFLOTRAN solves a system of generally nonlinear partial differential equations describing multiphase, multicomponent and multiscale reactive flow and transport in porous materials. The code is designed to run on massively parallel computing architectures as well as workstations and laptops (e.g. Hammond et al., 2011). Parallelization is achieved through domain decomposition using the PETSc (Port\-a\-ble Extensible Toolkit for Scientific Computation) libraries for the parallelization framework (Balay et al., 1997). 

PFLOTRAN has been developed from the ground up for parallel scalability and has been run on up to $2^{18}$ processor cores with problem sizes up to 2 billion degrees of freedom. Written in object oriented Fortran 90, the code requires the latest compilers compatible with Fortran 2003. At the time of this writing this requires gcc 4.7.x, Intel 12.1.x and PGC compilers. As a requirement of running problems with a large number of degrees of freedom, PFLOTRAN allows reading input data that is too large to fit into memory allotted to a single processor core. The current limitation to the problem size PFLOTRAN can handle is the limitation of the HDF5 file format used for parallel IO to 32 bit integers. Noting that $2^{32} = 4,294,967,296$, this gives an estimate of the maximum problem size that can be currently run with PFLOTRAN. Hopefully this limitation will be remedied in the near future.

Currently PFLOTRAN can handle a number of surface and subsurface processes including Richards equation, two-phase flow involving supercritical CO$_2$, and multicomponent reactive transport including aqueous complexing, sorption and mineral precipitation and dissolution. The reactive transport equations can be solved using either a fully implicit Newton-Raphson algorithm or the less robust operator splitting method. In addition, a novel approach is used to solve equations resulting from a multiple interacting continuum method for modeling flow and transport in fractured media. This implementation is still under development.

A novel feature of the code is its ability to run multiple input files and multiple realizations of permeability and porosity fields simultaneously on one or more processor cores per run. This can be extremely useful when conducting sensitivity analyses and quantifying model uncertainties. When running on machines with many cores this means that hundreds of simulations can be conducted in the amount of time needed for a single realization.

Additional instructions can be found on the PFLOTRAN \href{https://bitbucket.org/pflotran/pflotran-dev/wiki/Home}{wiki home page}.
Questions regarding the PFLOTRAN installation, and bug reports may be directed to: \linebreak {\footnotesize\tt pflotran-dev at googlegroups dot com}.
For questions regarding running PFLOTRAN contact: {\footnotesize\tt pflotran-users at googlegroups dot com}.
